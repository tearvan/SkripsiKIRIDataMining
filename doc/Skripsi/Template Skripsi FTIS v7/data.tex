%_____________________________________________________________________________
%=============================================================================
% data.tex v5 (10-11-2013) \ldots dibuat oleh Lionov - Informatika FTIS UNPAR
%
% Perubahan pada versi 5 (10-11-2013)
% - Perbaikan pada memasukkan bab : tidak perlu menuliskan apapun untuk 
%	memasukkan seluruh bab (bagian V)
% - Perbaikan pada memasukkan lampiran : tidak perlu menuliskan apapun untuk
%	memasukkan seluruh lampiran atau -1 jika tidak memasukkan apapun
%
% Perubahan pada versi sebelumnya
%	versi 4 (21-10-2012)
%	- Data dosen dipindah ke dosen.tex agar jika ada perubahan/update data dosen
%   mahasiswa tidak perlu mengubah data.tex
%	- Perubahan pada keterangan dosen	
% 	versi 3 (06-08-2012)
% 	- Perubahan pada beberapa keterangan 
% 	versi 2 (09-07-2012):
% 	- Menambahkan data judul dalam bahasa inggris
% 	- Membuat bagian khusus untuk judul (bagian VIII)
% 	- Perbaikan pada gelar dosen
%_____________________________________________________________________________
%=============================================================================
% 								BAGIAN -
%=============================================================================
% Ini adalah file data (data.tex)
% Masukkan ke dalam file ini, data-data yang diperlukan oleh template ini
% Cara memasukkan data dijelaskan di setiap bagian
% Data yang WAJIB dan HARUS diisi dengan baik dan benar adalah SELURUHNYA !!
%=============================================================================
%_____________________________________________________________________________
%=============================================================================
% 								BAGIAN I
%=============================================================================
% Tambahkan package2 lain yang anda butuhkan di sini
%=============================================================================
\usepackage{booktabs}
\usepackage[table]{xcolor}
\usepackage{longtable}
\usepackage{amsmath}
%=============================================================================

%_____________________________________________________________________________
%=============================================================================
% 								BAGIAN II
%=============================================================================
% Mode dokumen: menetukan halaman depan dari dokumen, apakah harus mengandung 
% prakata/pernyataan/abstrak dll (termasuk daftar gambar/tabel/isi) ?
% - kosong : tidak ada halaman depan sama sekali (untuk dokumen yang 
%            dipergunakan pada proses bimbingan)
% - cover : cover saja tanpa daftar isi, gambar dan tabel
% - sidang : cover, daftar isi, gambar, tabel (IT: UTS-UAS Seminar 
%			 dan UTS TA)
% - sidang_akhir : mode sidang + abstrak + abstract
% - final : seluruh halaman awal dokumen (untuk cetak final)
% Jika tidak ingin mencetak daftar tabel/gambar (misalkan karena tidak ada 
% isinya), edit manual di baris 439 dan 440 pada file main.tex
%=============================================================================
% \mode{kosong}
% \mode{cover}
% \mode{sidang}
%\mode{sidang_akhir}
\mode{final} 
%=============================================================================

%_____________________________________________________________________________
%=============================================================================
% 								BAGIAN III
%=============================================================================
% Line numbering: penomoran setiap baris, otomatis di-reset setiap berganti
% halaman
% - yes: setiap baris diberi nomor
% - no : baris tidak diberi nomor, otomatis untuk mode final
%=============================================================================
\linenumber{yes}
%=============================================================================

%_____________________________________________________________________________
%=============================================================================
% 								BAGIAN IV
%=============================================================================
% Linespacing: jarak antara baris 
% - single: opsi yang disediakan untuk bimbingan, jika pembimbing tidak
%            keberatan (untuk menghemat kertas)
% - onehalf: default dan wajib (dan otomatis) jika ingin mencetak dokumen
%            final/untuk sidang.
% - double : jarak yang lebih lebar lagi, jika pembimbing berniat memberi 
%            catatan yg banyak di antara baris (dianjurkan untuk bimbingan)
%=============================================================================
%\linespacing{single}
 \linespacing{onehalf}
%\linespacing{double}
%=============================================================================

%_____________________________________________________________________________
%=============================================================================
% 								BAGIAN V
%=============================================================================
% Bab yang akan dicetak: isi dengan angka 1,2,3 s.d 9, sehingga bisa digunakan
% untuk mencetak hanya 1 atau beberapa bab saja
% Jika lebih dari 1 bab, pisahkan dengan ',', bab akan dicetak terurut sesuai 
% urutan bab.
% Untuk mencetak seluruh bab, kosongkan parameter (i.e. \bab{ })  
% Catatan: Jika ingin menambahkan bab ke-10 dan seterusnya, harus dilakukan 
% secara manual
%=============================================================================
\bab{ }
%=============================================================================

%_____________________________________________________________________________
%=============================================================================
% 								BAGIAN VI
%=============================================================================
% Lampiran yang akan dicetak: isi dengan huruf A,B,C s.d I, sehingga bisa 
% digunakan untuk mencetak hanya 1 atau beberapa lampiran saja
% Jika lebih dari 1 lampiran, pisahkan dengan ',', lampiran akan dicetak 
% terurut sesuai urutan lampiran
% Jika tidak ingin mencetak lampiran apapun, isi dengan -1 (i.e. \lampiran{-1})
% Untuk mencetak seluruh mapiran, kosongkan parameter (i.e. \lampiran{ })  
% Catatan: Jika ingin menambahkan lampiran ke-J dan seterusnya, harus 
% dilakukan secara manual
%=============================================================================
\lampiran{ }
%=============================================================================

%_____________________________________________________________________________
%=============================================================================
% 								BAGIAN VII
%=============================================================================
% Data diri dan skripsi/tugas akhir
% - namanpm: Nama dan NPM anda, penggunaan huruf besar untuk nama harus benar
%			 dan gunakan 10 digit npm UNPAR, PASTIKAN BAHWA BENAR !!!
%			 (e.g. \namanpm{Jane Doe}{1992710001}
% - judul : Dalam bahasa Indonesia, perhatikan penggunaan huruf besar, judul
%			tidak menggunakan huruf besar seluruhnya !!! 
% - tanggal : isi dengan {tangga}{bulan}{tahun} dalam angka numerik, jangan 
%			  menuliskan kata (e.g. AGUSTUS) dalam isian bulan
%			  Tanggal ini adalah tanggal dimana anda akan melaksanakan sidang 
%			  ujian akhir skripsi/tugas akhir
% - pembimbing: isi dengan pembimbing anda, lihat daftar dosen di file dosen.tex
%				jika pembimbing hanya 1, kosongkan parameter kedua 
%				(e.g. \pembimbing{\JND}{  } )
% - penguji : isi dengan para penguji anda, lihat daftar dosen di file dosen.tex
%				(e.g. \penguji{\JHD}{\JCD} )
%=============================================================================
\namanpm{Jovan Gunawan}{2011730029}
\tanggal{14}{9}{2014}
\pembimbing{\PAS}{}     %Lihat singkatan pembimbing anda di file dosen.tex
%\penguji{\NIS}{\CEN} 		%Lihat singkatan penguji anda di file dosen.tex
%=============================================================================

%_____________________________________________________________________________
%=============================================================================
% 								BAGIAN VIII
%=============================================================================
% Judul dan title : judul bhs indonesia dan inggris
% - judulINA: judul dalam bahasa indonesia
% - judulENG: title in english
% PERHATIAN: - langsung mulai setelah '{' awal, jangan mulai menulis di baris 
%			   bawahnya
%			 - Gunakan \texorpdfstring{\\}{} untuk pindah ke baris baru
%			 - Judul TIDAK ditulis dengan menggunakan huruf besar seluruhnya !!
%=============================================================================

\judulINA{Data Mining Histori Pencarian Rute Angkot}

\judulENG{Data Mining History searching route}

%_____________________________________________________________________________
%=============================================================================
% 								BAGIAN IX
%=============================================================================
% Abstrak dan abstract : abstrak bhs indonesia dan inggris
% - abstrakINA: abstrak bahasa indonesia
% - abstrakENG: abstract in english
% PERHATIAN: langsung mulai setelah '{' awal, jangan mulai menulis di baris 
%			 bawahnya
%=============================================================================

\abstrakINA{Dengan kemajuan teknologi informasi saat ini, kebutuhan akan informasi yang akurat dibutuhkan dalam kehidupan sehari-hari, namun informasi tersebut perlu diolah agar dapat disajikan dengan baik. 
Jika dilihat lebih rinci, data tersebut dapat diolah lebih lanjut dan menghasilkan sesuatu yang lebih. \textsl{Data mining} merupakan salah satu proses untuk mengolah data untuk mempermudah menganalisis dan mengambil keputusan dari data yang dimiliki. 
Salah satu teknik dari \textsl{data mining} adalah dengan membuat \textsl{decision tree} untuk melakukan klasifikasi suatu objek. 
Terdapat beberapa metode untuk memilih atribut pada pembuatan \textsl{decision tree}, dua diantaranya adalah ID3 dan C4.5. Metode ID3 merupakan metode yang menggunakan nilai \textsl{entropy} untuk memilih atribut sedangkan C4.5 menggunakan nilai \textsl{entropy} dan \textsl{gain ratio} untuk memilih atribut. 

Pada penelitian ini, percobaan \textsl{data mining} dilakukan pada data \textsl{log} histori KIRI bulan Febuari 2014, khususnya untuk \textsl{action} dengan nilai FINDROUTE. 
Atribut yang akan diuji adalah \textsl{timestamp} dan additionalData yang berisi lokasi keberangkatan dan tujuan dari \textsl{user} yang menggunakan aplikasi KIRI. Pada percobaan ini, akan dibuat klasifikasi sepuluh daerah di Bandung berdasarkan titik pusat Bandung dengan radius satu kilometer untuk setiap daerah.
Dengan klasifikasi tersebut, dapat ditentukan apakah \textsl{user} menjauhi Bandung atau mendekati Bandung atau menuju daerah yang sama. Pembuatan \textsl{decision tree} digunakan untuk melakukan klasifikasi apakah pada hari tertentu dan jam tertentu, \textsl{user} akan lebih sering menjauhi Bandung atau menuju Bandung atau menuju daerah yang sama. 
Dari hasil pengujian experimental, diperoleh bahwa \textsl{decision tree} yang dibuat dengan ID3 mengalami overfitting dengan akurasi 38.22\%, sedangkan dengan J48 (metode C4.5 dari weka) tidak mengalami overfitting dengan akurasi 50.37\% dan diperoleh kesimpulan bahwa \textsl{user} sering menjauhi Bandung daripada menuju Bandung atau menuju daerah yang sama pada bulan Febuari 2014.
}
%Dari hasil pengujian experimental, diperoleh bahwa \textsl{decision tree} yang dibuat dengan ID3 mengalami overfitting dengan akurasi 37.49\%, sedangkan dengan J48 (metode C4.5 dari weka) tidak mengalami overfitting dengan akurasi 49.79\% dan diperoleh kesimpulan bahwa \textsl{user} sering menjauhi Bandung dan menuju Bandung, namun jarang atau tidak pernah menuju daerah yang sama.

\abstrakENG{With the advancement of technology these day, accurate information is needed everyday, but the information need to be treated in order to be presented better.
If viewed in more detail, the data can be treated furthermore and produce something better. \textsl{Data mining} is one of process to process data to ease analyze and draw conclusions.
One of the techniques of \textsl{data mining} is to make \textsl{decision tree} to classify object.
There are several methods to choose attribute at \textsl{decision tree}, two of which are ID3 and C4.5. ID3 method is a method that uses \textsl{entropy} to choose attribute, while C4.5 uses \textsl{entropy} and \textsl{gain ratio} to do that.

In this study, \textsl{data mining} experiments performed on the log KIRI history in February 2014, in particular to the action with the value FINDROUTE.
Attributes to be tested is timestamp and additionalData which contains departure location and destination location from user who uses KIRI application. In this experiment, will be made ten regional classification in Bandung based on center point of Bandung with radius one kilometer for each region.
With the regional classification, can be determined whether the user away from Bandung or heading to Bandung or heading to same region. Making the \textsl{decision tree} is used to classify whether on certain days and certain hours, users will be more frequent to away Bandung or heading to Bandung or heading to same region.
From the results of experimental testing, obtained that the \textsl{decision tree} created with ID3 is overfitting with 38.22\% accuracy, while J48 (method C4.5 from weka) is not overfitting with 50.37\% accuracy, and it is concluded that user more often away from Bandung than heading Bandung or heading to same region at February 2014.
} 

%From the results of experimental testing, obtained that the \textsl{decision tree} created with ID3 is overfitting with 39.49\% accuracy, while J48 (method C4.5 from weka) is not overfitting with 49.79\% accuracy, and it is concluded that user often away from Bandung and heading Bandung, but rarely or never heading to same region.} 

%=============================================================================

%_____________________________________________________________________________
%=============================================================================
% 								BAGIAN X
%=============================================================================
% Kata-kata kunci dan keywords : diletakkan di bawah abstrak (ina dan eng)
% - kunciINA: kata-kata kunci dalam bahasa indonesia
% - kunciENG: keywords in english
%=============================================================================
\kunciINA{Data Mining, Decision Tree, KIRI}

\kunciENG{Data Mining, Decision Tree, KIRI}
%=============================================================================

%_____________________________________________________________________________
%=============================================================================
% 								BAGIAN XI
%=============================================================================
% Persembahan : kepada siapa anda mempersembahkan skripsi ini ...
%=============================================================================
\untuk{Dipersembahkan untuk diri sendiri }
%=============================================================================

%_____________________________________________________________________________
%=============================================================================
% 								BAGIAN XII
%=============================================================================
% Kata Pengantar: tempat anda menuliskan kata pengantar dan ucapan terima 
% kasih kepada yang telah membantu anda bla bla bla ....  
%=============================================================================
\prakata{Puji syukur kepada Tuhan yang Maha Esa atas petunjuk dan rahmat-Nya, penulis dapat menyelesaikan laporan skripsi tanpa ada halangan apapun dan dapat selesai tepat pada waktunya.

Laporan skripsi yang telah saya susun ini dibuat dalam rangka memenuhi salah satu syarat untuk mendapatkan gelar sarjana (S1) Program Studi Ilmu Komputer, Fakultas Teknologi Informasi dan Sains, Universitas Katolik Parahyangan.

Saya menyadari bahwa skripsi ini dapat diselesaikan karena peranan banyak pihak. Oleh karena itu, saya ingin mengucapkan terima kasih sebesar-besarnya kepada:

\begin{enumerate}
	\item Papa, Mama, Koko, dan Cici yang telah memberikan dukungan dalam mengerjakan skripsi ini pada saya.
	\item Pak Pascal selaku dosen wali atas bimbingannya selama saya mengerjakan skripsi ini dan selaku developer KIRI yang telah mengizinkan saya untuk melakukan \textsl{data mining} serta analisis pada data \textsl{log} histori KIRI.
	\item Dosen-dosen penguji yang telah meluangkan waktu untuk mempelajari skripsi ini dan memberikan kritik dan saran.
	\item Pak Lionov yang telah memberikan informasi, tips, serta dorongan dalam melaksanakan skripsi.
	\item Clara, Antonio, Steven, Samuel, Kevin, dan seluruh angkatan 2011 IT Universitas Parahyangan atas segala dukungan, bantuan, saran, persahabatan, serta kerja samanya selama ini.
\end{enumerate}
Serta seluruh pihak yang tidak dapat disebutkan satu-persatu atas dukungan kalian yang sangat bearti bagi saya, Tuhan memberkati.

Saya menyadari bahwa skripsi ini masih jauh dari sempurna, oleh karena itu, saya menerima saran dan kritik agar bisa menghasilkan yang lebih baik. Akhir kata, saya berharap skripsi ini dapat bermanfaat bagi perkembangan ilmu teknologi khususnya di bidang \textsl{data mining}, dan bagi KIRI serta para pembaca.}

%=============================================================================

%_____________________________________________________________________________
%=============================================================================
% 								BAGIAN XIII
%=============================================================================
% Tambahkan hyphen (pemenggalan kata) yang anda butuhkan di sini 
%=============================================================================
\hyphenation{ma-te-ma-ti-ka}
\hyphenation{fi-si-ka}
\hyphenation{tek-nik}
\hyphenation{in-for-ma-ti-ka}
%=============================================================================


%=============================================================================
