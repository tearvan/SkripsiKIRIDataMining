\chapter{Landasan Teori}
\label{chap:definition}

\section{\textsl{Knowledge Discovery}}

\textsl{Knowledge Discovery} atau yang sering disebut juga \textsl{Data mining}, merupakan merupakan proses pengambilan inti sari atau penggalian \textsl{knowledge} dari data yang besar. 


\begin{figure}[H]
\centering
\includegraphics[scale=0.9]{Gambar/tahapdatamining.jpg}
\caption[Tahap \textsl{Data Mining}]{Tahap \textsl{Data Mining}\cite{DM}} 
\label{fig:tahapDataMining}
\end{figure}

Menurut \cite{DM}, \textsl{Knowledge Discovery} dapat dibagi menjadi 7 tahap (gambar \ref{fig:tahapDataMining}):
\begin{enumerate}
	\item \textsl{Data cleaning}
	\item \textsl{Data integration}
	\item \textsl{Data selection}
	\item \textsl{Data transformation}
	\item \textsl {Data mining}
	\item \textsl{Pattern Evaluation}
	\item \textsl{Knowledge presentation}
\end{enumerate}

Tahap pertama hingga keempat merupakan bagian dari \textsl{data preprocessing}, dimana data-data disiapkan untuk dilakukan penggalian data. Tahap \textsl{data mining} merupakan tahap dimana penggalian data dilakukan. Tahap keenam merupakan tahap pencarian pola yang merepresentasikan \textsl{knowledge}. Sedangkan tahap terakhir merupakan visualisasi dan representasi dari \textsl{knowledge} yang sudah diperoleh dari tahap sebelumnya.

\subsection{\textsl{Data Cleaning}}
\textsl{Data cleaning} merupakan tahap \textsl{Knowledge Discovery} untuk menghilangkan \textsl{missing value} dan \textsl{noisy data}. \textsl{Missing value} merupakan nilai yang hilang dari suatu data. \textsl{Noisy data} merupakan nilai yang tidak sesuai atau tidak valid.

Pada umumnya, \textsl{data} yang diperoleh dari \textsl{database} terdapat nilai yang tidak sempurna seperti nilai yang hilang, nilai yang tidak valid atau salah ketik. Atribut dari suatu \textsl{database} yang tidak relevan atau redudansi bisa diatasi dengan menghapus atribut atau tuple tersebut. Contoh data yang memiliki \textsl{missing value} dan \textsl{noisy data} dapat dilihat pada table \ref{table:contohMissingNNoisy}

\begin{table}[h]
\centering
\caption{table mengandung \textsl{missing value} dan \textsl{noisy data}}
\label{table:contohMissingNNoisy}
\begin{tabular}{|l|l|l|l|l|}
\hline
IdPenjualan & NamaBarang & Customer & Harga  & BanyakBarang \\ \hline
1           & Mouse      & Elvin    & 45000  & 2            \\ \hline
2           & Keyboard   & Alleria  & -35000 & 1            \\ \hline
3           & Monitor    &          & 225000 & 1            \\ \hline
\end{tabular}
\end{table}

Dapat dilihat, pada idPenjualan 2, harga dari keyboard adalah -35000, itu merupakan \textsl{noisy data} karena tidak mungkin nilai harga suatu barang dibawah 0. Pada idPenjualan 3, kolom \textsl{customer} tidak memiliki nilai, dan itu merupakan \textsl{missing value}.

\subsubsection{\textsl{Missing Values}}
\textsl{Missing values} akan mengganggu proses \textsl{data mining} pada komputer dan dapat menghasilkan nilai akhir yang tidak sesuai dengan fakta. Terdapat beberapa teknik untuk mengatasi \textsl{missing values} yaitu:
	\begin{itemize}
		\item Membuang tuple yang mengandung nilai yang hilang\textit{\textit{}}
		\item Mengisi nilai yang hilang secara manual
		\item Mengisi nilai yang hilang dengan menggunakan nilai konstan yang bersifat umum
		\item Menggunakan nilai rata-rata dari suatu atribut untuk mengisi nilai yang hilang
	\end{itemize}
\subsubsection{\textsl{Noisy Data}}
\textsl{Noisy data} merupakan nilai yang berasal dari error atau tidak valid. \textsl{Noisy data} dapat dihilangkan dengan menggunakan teknik \textsl{smoothing}. Teknik \textsl{smoothing} merupakan teknik untuk menghilangkan \textsl{noisy data}. Terdapat 3 metode untuk menghilangkan \textsl{noisy data} yaitu:
	\label{teknikBinning}
	\begin{itemize}
		\item \textsl{Binning}, merupakan metode pengisian data dengan proses yang dilakukan pada data tersebut
		\item \textsl{Regression}, merupakan metode yang mencari detil persamaan atribut untuk memprediksikan suatu nilai
		\item	\textsl{Clustering}, merupakan metode pengelompokan dimana ditemukan pencilan yang dapat dibuang. Pencilan merupakan data yang berada diluar kelompok yang sesuai dengan observasi atau penelitian.
	\end{itemize}

%\subsubsection{\textsl{Data Cleaning as a Process}}
%Tahap pertama pada \textsl{data cleaning} adalah \textsl{discrepancy detection}. Ketidakcocokan dapat dikarenakan oleh beberapa faktor, termasuk desain data yang buruk, kesalahan manusia ketika memasukan data, dan data yang sudah kadarluarsa. Ketidakcocokan ini juga dapat disebabkan tidak konsisten representasi data dan kode atau dikarenakan kesalahan perangkat ketika melakukan pemasukan data.

%Untuk mempermudah pencarian ketidakcocokan tersebut, kita dapat membuat sebuah data yang berisi informasi mengenai data atau biasa disebut \textsl{metadata}. Pada tahap ini, penulisan \textsl{script} bisa ditulis dengan cara masing-masing. Disini, dapat ditemukan \textsl{noise}, \textsl{outliers}, dan nilai-nilai yang tidak cocok atau tidak konsisten.

%Data juga harus diperiksa dengan \textsl{unique rules}, \textsl{consecutive rules}, dan \textsl{null rules}. \textsl{unique rules} mengatakan bahwa setiap nilai dari sebuah atribut harus berbeda dengan nilai yang lain pada atribut tersebut. \textsl{Consecutive rules} mengatakan bahwa tidak boleh ada nilai yang hilang diantara nilai tertinggi dan terendah untuk sebuah atribut, dan semua nilai harus bersifat unik. \textsl{null rules} menspesifikasikan penggunaan nilai \textsl{blanks} atau kosong, tanda tanya, karakter spesial, atau \textsl{string} yang dapat menandakan bahwa nilai tersebut bersifat kosong.

%Tahap deteksi ketidakcocokan ini dapat dibantu juga dengan menggunakan \textbf{data scrubbing tools} dan \textbf{data auditing tools}. \textbf{data scrubbing tools} akan menggunakan sebuah domain data untuk melakukan pencarian ketidakcocokan data dan membetulkan data tersebut dengan menggunakan teknik \textsl{parsing} dan \textsl{fuzzy matching}. \textbf{Data auditing tools} akan mencari ketidakcocokan dengan melakukan analisa data untuk menemukan \textsl{rules}, relasi, dan mendeteksi data yang melanggar hal tersebut.

%Beberapa data dapat diperbaiki dengan cara manual, namun sebagian besar data akan membutuhkan \textsl{data transformation} untuk membetulkan data tersebut.

\subsection{\textsl{Data Integration}}
\textsl{Data integration} merupakan tahap penggabungan data dari berbagai sumber. Sumber tersebut bisa termasuk beberapa \textsl{database}, \textsl{data cubes}, atau \textsl{flat data}. \textsl{Data cube} merupakan teknik pengambilan data-data dari \textsl{data warehouse} dan dilakukan operasi agregasi sesuai dengan kondisi tertentu (contoh, penjumlahan total dari penjualan per tahun dari 2005-2010). Sedangkan \textsl{flat data} merupakan data yang disimpan dengan cara apapun untuk merepresentasikan model database pada sebuah data baik berbentuk \textsl{plain text file} maupun \textsl{binary file}. 

Tahap ini harus dilakukan secara teliti terutama dalam memasangkan nilai-nilai yang berasal dari sumber yang berbeda. Pada tahap ini, perlu dilakukan identifikasi apakah data tertentu harus dimasukkan atau tidak agar data yang diperoleh tidak terlalu besar. \textsl{Data integration} yang baik merupakan integrasi yang dapat memaksimalkan kecepatan dan meningkatkan akurasi dari proses \textsl{data mining}. Contoh studi kasus dari \textsl{data integration}, misalnya suatu perusahaan sepatu A memiliki dua pabrik dengan \textsl{database} lokal pada masing-masing pabrik. Jika akan dilakukan \textsl{data mining} pada kedua \textsl{database }tersebut, maka kedua \textsl{database} harus digabung. Ketika digabung, harus memperhatikan dan memperbaiki nilai-nilai seperti \textsl{primary key}, atribut. Hal ini dilakukan untuk menghindari kesalahan seperti nilai id yang berbeda padahal merupakan objek yang sama. Proses penggabungan hingga perbaikan nilai-nilai pada kedua database tersebut disebut proses \textsl{data integration}.

\subsection{\textsl{Data Selection}}
Proses dimana data-data yang relevan dengan analisis akan diambil dari database dan data yang tidak relevan akan dibuang. Sebagai contoh kasus, jika akan dilakukan analisa mengenai nilai mahasiswa pada table nilai yang memiliki atribut sebagai berikut:
	\begin{itemize}
		\item NPMMahasiswa
		\item NamaMahasiswa
		\item JenisKelamin
		\item Alamat
		\item MataKuliah
		\item NilaiART
		\item NilaiUTS
		\item NilaiUAS
	\end{itemize}
Maka, atribut yang berpotensi diambil adalah MataKuliah, NilaiART, NilaiUTS dan NilaiUAS, sedangkan atribut yang dibuang adalah NPMMahasiswa, NamaMahasiswa JenisKelamin, dan Alamat karena tidak  berhubungan dengan analisa.

\subsection{\textsl{Data Transformation}}
\textsl{Data transformation} merupakan tahap pengubahan data agar siap dilakukan proses \textsl{data mining}. \textsl{Data transformation} bisa melibatkan:
	\begin{itemize}
		\item \textsl{Smoothing}, proses untuk membuang \textsl{noise} seperti yang dilakukan pada tahap \textsl{data cleaning}
		\item \textsl{Aggregation}, proses merangkum nilai-nilai menjadi suatu nilai yang dapat mewakili nilai sebelumnya
		\item \textsl{Generalization}, proses membuat suatu nilai yang bersifat khusus menjadi nilai yang bersifat umum
		\item \textsl{Normalization}, proses dimana suatu nilai dapat diubah skalanya menjadi nilai yang lebih kecil dan spesifik
		\item \textsl{Attribute construction}, proses membuat atribut baru yang berasal dari beberapa atribut untuk membantu proses data mining
	\end{itemize}
	
\subsubsection{\textsl{Smoothing}}
\textsl{Smoothing} merupakan teknik untuk menghilangkan \textsl{noise} pada database. Teknik dari \textsl{smoothing} adalah \textsl{binning}, \textsl{regression}, dan \textsl{clustering}. Penjelasan teknik \textsl{smoothing} dapat dilihat pada \ref{teknikBinning}, bagian \textsl{noisy data}.

\subsubsection{\textsl{Aggregation}}
\textsl{Aggregation} merupakan teknik melakukan operasi agregat untuk mendapatkan nilai yang digunakan di tahap \textsl{data mining}. Contoh kasus, jika terdapat suatu database dari toko A, dapat menggunakan operasi agregat untuk mencari total pendapatan dengan rentang hari tertentu.

\subsubsection{\textsl{Generalization}}	
\textsl{generalization} merupakan teknik mengubah data yang bersifat \textsl{primitive} atau \textsl{low level} menjadi \textsl{high level} dengan menggunakan konsep hirarki. Contoh kasus, nilai pada atribut umur dapat dikelompokkan menjadi muda, dewasa, tua.	
	
\subsubsection{\textsl{Normalization}}
\textsl{Normalisasi} merupakan teknik mengubah nilai atribut menjadi nilai baru yang memiliki range yang lebih spesifik dan kecil seperti 0,0 sampai 1,0.
Terdapat beberapa teknik normalisasi, dua diantaranya yaitu, \textsl{min-max normalization} dan \textsl{z-score normalization}. \textsl{Min-max normalization} akan mengubah semua nilai menjadi nilai dengan skala tertentu. Rumus dari teknik \textsl{min-max normalization} sebagai berikut

\begin{displaymath}
	\nu' = \frac{\nu-min_{A}}{max_{A}-min_{A}}(newMax_{A}-newMin_{A})+newMin_{A}	
\end{displaymath}

Contoh kasus, misalkan nilai minimum dan maximum dari suatu pendapatan adalah 12.000 dan 98.000, akan diubah menjadi berskala antara 0,0 sampai 1,0. Jika ada nilai pendapat yang baru, yaitu 73.600, maka akan menjadi

\begin{displaymath}
\frac{73.600-12.000}{98.000-12.000} (1,0-0)+0 = 0,716
\end{displaymath}

\textsl{z-score normalization} merupakan mengubah nilai berdasarkan rata-rata dan standar deviasi dari atribut. Rumus dari \textsl{z-score normalization} sebagai berikut

\begin{displaymath}
\nu' = \frac{\nu-\overline{A}}{\sigma_{A}}
\end{displaymath}

Contoh kasus, misal nilai rata-rata dan standar deviasi dari nilai-nilai atribut pendapatan adalah 54.000 dan 16.000. Dengan \textsl{z-score}, jika ada nilai pendapatan baru yaitu 73600, maka akan diubah menjadi

\begin{displaymath}
\frac{73.600-54.000}{16.000} = 1,225 
\end{displaymath}

\subsubsection{\textsl{Attribute Construction}}
\textsl{Attribute Construction} merupakan teknik menambahkan atribut baru yang berdasarkan atribut yang sudah ada. Contoh kasus, dibuat atribut baru bernama area berdasarkan atribut panjang dan lebar. 

%\subsubsection{\textsl{Data Reduction}}
%Proses \textsl{aggregation} dan \textsl{generalization} akan dilakukan dalam bentuk proses \textsl{data reduction} dan \textsl{Data Cube Aggregation}.
%\textsl{Data reduction} dan dilakukan untuk mendapatkan nilai yang representif namun tetap menjaga keakuratan hasil \textsl{data mining}. Terdapat beberapa cara dalam mengimplementasikan \textsl{data reduction} yaitu
%	\begin{itemize}
%		\item \textsl{Data subset selection}
%		\item \textsl{Dimensionality reduction}
%		\item \textsl{Numerosity reduction}
%		\item \textsl{Discretization and concept hierarchy generation}
%	\end{itemize}	

%\subsubsection {\textsl{Attribute Subset Selection}}
%\textsl{Attribute subset selection} merupakan salah satu cara melakukan \textsl{data reduction} dengan menghilangkan atribut-atribut yang tidak relevan atau data yang redudansi. Hal ini dapat mempermudah pencarian pola dikarenakan atribut yang tidak relevan tidak ada. Tujuan dari \textsl{attribute subset selection} adalah memperoleh set data yang paling sedikit yang tetap menghasilkan probabilitas penyebaran data kelas tetap mirip dengan set data sebelum dikurangi atributnya. 
%Contoh studi kasus, jika toko CD ingin mencari apakah para konsumen tertarik dan membeli CD baru, maka atribut nama dan nomor telepon dari data konsumen tidak terlalu relevan dengan tujuan \textsl{mining} pada kasus ini, sedangkan jenis\_kelamin dan musik\_kesukaan dari konsumen bisa menjadi atribut yang relevan. 

%\subsubsection {\textsl{Dimensionality Reduction}}
%\textsl{Dimensionality Reduction} merupakan metode pengurangan nilai secara acak dengan cara melakukan konversi data. Jika data original dapat dibuat ulang dari data yang sudah dikompresi tanpa kehilangan informasi, maka akan dikatakan \textsl{lossless}, namun jika hanya mendapatkan data pendekatannya saja, akan disebut lossly \cite{DM}.

%\subsubsection {\textsl{Numerosity Reduction}}
%\textsl{Numerosity Reduction} merupakan metode dimana data diganti atau ditentukan dengan cara parametik atau nonparametrik.

%\subsubsection {\textsl{Discretization and Concept Hierarchy Generation}}
%lewat dulu

\subsection{\textsl{Data Mining}}

Pada tahap ini, akan dilakukan proses \textsl{data mining} dengan menggunakan input data yang sudah diproses pada tahap sebelumnya (\textsl{data cleaning, data selection, data integration,} dan \textsl{data transformation}).

\subsubsection{\textsl{Classification and Prediction}}
\textsl{Classification} merupakan pemodelan yang dibangun untuk memprediksikan label kategori. Contoh label kategori adalah "baik", "cukup", dan "buruk" dalam sistem penilaian sikap seorang siswa atau "mini bus", "bus", atau "sedan" dalam kategori tipe mobil. Kategori dapat direpresentasikan dengan menggunakan nilai diskret. Nilai diskret merupakan nilai yang terpisah dan berbeda. Contoh dari nilai diskret adalah 1 atau 5. Kategori yang direpresentasikan oleh nilai diskret maka akan menjadi nilai yang terurut dan tidak memiliki arti. Contoh kategori yang direpresentasikan oleh nilai diskret adalah 1,2,3. Angka tersebut dapat digunakan untuk merepresentasikan suatu kategori, misalnya untuk tipe mobil: 
\begin{itemize}
	\item Angka 1 adalah "mini bus"
	\item Angka 2 adalah "bus"
	\item Angka 3 adalah "sedan"
\end{itemize}.

\textsl{Prediction} merupakan model yang dibangun untuk meramalkan fungsi nilai kontinu. Nilai kontinu merupakan nilai yang terurut dan berlanjut. Contoh kasus untuk pemodelan prediction, misalkan seorang marketing ingin meramalkan seberapa banyak konsumen yang akan belanja di sebuah toko dalam waktu satu bulan. Pemodelan tersebut disebut \textsl{predictor}. \textsl{Regression Analysis} merupakan metodologi statistik yang digunakan untuk \textsl{numeric prediction}. \textsl{Classification} dan \textsl{numeric prediction} merupakan dua fungsi utama pada \textsl{prediction}.

\textsl{Data Classification} merupakan proses untuk melakukan klasifikasi label kategori. \textsl{Data classification} memiliki dua tahap proses, yaitu \textsl{learning step} dan tahap klasifikasi. \textsl{Learning step} merupakan langkah pembelajaran, di mana algoritma klasifikasi membangun \textsl{classification rules} (yang berisi syarat atau aturan sebuah nilai masuk ke dalam kategori tertentu) dengan cara menganalisis \textsl{training set} yang merupakan \textsl{database tuple}. Karena pembuatan \textsl{classification rules} menggunakan \textsl{training set}, yang dikenal juga sebagai \textsl{supervised learning}. Pada tahap kedua, dilakukan proses klasifikasi nilai berdasarkan \textsl{classification rules} yang sudah dibangun dari tahap pertama.

Contoh kasus \textsl{data classification} dapat dilihat pada ilustrasi di gambar \ref{fig:tahapDataClassification}. Pada gambar a, data pelatihan akan diproses oleh algoritma klasifikasi dan menghasilkan aturan klasifikasi. Aturan tersebut akan digunakan untuk menentukan label kategori. Pada gambar b, data uji akan diproses oleh aturan klasifikasi yang sudah diperoleh dari gambar a dan akan menghasilkan hasil prediksi bahwa suatu tuple beresiko akan peminjaman kredit atau tidak.

\begin{figure}
\centering
\includegraphics[scale=0.8]{Gambar/tahapdataclassification.jpg}
\caption[Tahap \textsl{data classification}]{Tahap \textsl{data classification}, Diterjemahkan dari \cite{DM}} 
\label{fig:tahapDataClassification}
\end{figure}

\subsubsection{\textsl{Decision Tree}}
Salah satu cara pembuatan \textsl{classification rules} pada \textsl{Data Classification} adalah membangun \textsl{decision tree} (pohon keputusan). \textsl{Decision tree} merupakan \textsl{flowchart} yang berbentuk pohon, dimana setiap node internal (\textsl{nonleaf} node) merupakan hasil test dari atribut, setiap cabang merepresentasikan output dari test, dan setiap node daun memiliki \textsl{class label}. Bagian paling atas dari pohon disebut \textsl{root node}. 

Contoh kasus, pohon keputusan untuk menentukan apakah seorang konsumen akan membeli komputer atau tidak (ilustrasi pohon keputusan pada gambar \ref{fig:decisionTree}). Root node dari pohon keputusan adalah umur. Atribut pertama yang akan diperiksa adalah atribut umur. Jika nilai atribut umur dari suatu tuple adalah muda, maka akan diperiksa atribut murid apakah bernilai bukan atau ya. Jika nilai atribut murid adalah bukan, maka tuple tersebut memiliki label tidak membeli komputer sedangkan jika bernilai ya, maka tuple tersebut akan membeli komputer. Jika nilai atribut umur adalah remaja, maka tuple tersebut berlabel membeli komputer. Jika atribut umur bernilai dewasa maka akan diperiksa atribut peringkat kredit, apakah cukup baik atau baik. Jika atribut peringkat kredit bernilai cukup baik maka tuple tersebut memiliki label tidak membeli komputer sedangkan jika bernilai baik, maka tuple tersebut akan membeli komputer.

\begin{figure}
\centering
\includegraphics[scale=1]{Gambar/decisiontree.jpg}
\caption[Contoh \textsl{decision tree}]{Contoh \textsl{decision tree}, Diterjemahkan dari \cite{DM}} 
\label{fig:decisionTree}
\end{figure}

\paragraph{\textsl{Decision Tree Induction}} merupakan pelatihan pohon keputusan dari tupel pelatihan yang memiliki label kategori. Algoritma yang diperlukan secara umum sama, hanya berbeda pada \textsl{attribute\_selection\_method}. Berikut algoritma untuk membuat pohon keputusan dari suatu tupel pelatihan:


\begin{algorithmic}[1]
	\REQUIRE Partisi data, D, merupakan set data pelatihan dan kelas label
	\REQUIRE \textsl{attribute\_list}, merupakan set dari atribut kandidat
	\REQUIRE \textsl{Attribute\_selection\_method}, prosedur untuk menentukan \textsl{splitting criterion}. Pada input ini, terdapat juga data \textsl{splitting\_attribute} dan mungkin salah satu dari \textsl{split point} atau \textsl{splitting subset}
	\ENSURE Pohon keputusan
	\STATE Membuat node N;
	\IF{tuple pada D merupakan kelas yang sama, C} 
	  \RETURN N sebagai node daun dengan label kelas C;
	\ENDIF
	\IF{attribute\_list tidak ada nilai atau kosong}
		\RETURN N sebagai node daun dengan label kelas yang terpaling banyak pada D; \COMMENT {majority voting}
	\ENDIF
	\STATE memanggil method Attribute\_selection\_method(D, atribute\_list) untuk mencari nilai terbaik splitting\_criterion;
	\STATE menamakan node N dengan splitting\_criterion;
	\IF{splitting\_attribute merupakan nilai discrete and multiway splits diizinkan}
		\STATE attribute\_list $\leftarrow$ attribute\_list - splitting\_attribute; \COMMENT{menghapus splitting\_attribute}
	\ENDIF
	\FORALL{hasil j dari splitting\_criterion}
		\STATE D\lowercase{j} merupakan himpunan data tupel D yang sesuai dengan j;
		\IF{D\lowercase{j} tidak ada nilai atau kosong}
			\STATE melampirkan daun yang diberi label dengan kelas mayoritas di D ke node N;
		\ELSE
			\STATE melampirkan node yang dikembalikan oleh generate\_decision\_tree(D\lowercase{j}, attribute\_list) ke node N;
		\ENDIF
	\ENDFOR
\RETURN N;
\end{algorithmic}

Pohon keputusan akan dimulai dengan satu node, yaitu N, merepresentasikan tuple pelatihan pada D (baris 1)

Jika tuple di D memiliki kelas yang sama semua, maka node N akan menjadi daun dan diberi label dari kelas tersebut (baris 2 sampai 4). 

Jika tuple di D memiliki kelas yang berbeda, maka algoritma akan memanggil \textsl{attribute\_selection\_method} untuk menentukan \textsl{splitting criterion}. \textsl{Splitting criterion} akan menentukan atribut pada node N. (baris 8)

Node N akan diisi dengan hasil dari \textsl{splitting criterion} (baris 9). Kemudian kriteria tersebut membentuk cabangnya masing-masing sesuai pada baris 13 dan 14. Terdapat tiga kemungkinan bentuk kriteria jika A merupakan \textsl{splitting\_attribute} yang memiliki nilai unik seperti \{a$_{1}$, a$_{2}$, ..., a$_{v}$\}. Tiga kemungkinan tersebut dapat dilihat pada gambar \ref{fig:splitPoint}, berikut penjelasannya:

\begin{enumerate}
	\item \textbf{\textsl{Discrete valued}}: cabang yang dihasilkan memiliki kelas dengan nilai diskret. Karena kelas yang dihasilkan diskret dan hanya memiliki nilai yang sama pada cabang tersebut, maka \textbf{\textsl{attribut\_list}} akan dihapus (baris 10 sampai 12)
	\item \textbf{\textsl{Continuous values}}: cabang yang dihasilkan memiliki jarak nilai untuk memenuhi suatu kondisi (contoh: A <= split\_point), dimana nilai \textsl{split\_point} adalah nilai pembagi yang dikembalikan oleh \textsl{attribute\_selection\_method}
	\item \textbf{\textsl{Dicrete valued and a binary tree}}: cabang yang dihasilkan berupa nilai iya dan tidak dari "apakah A anggota S$_{a}$", dimana S$_{a}$ merupakan subset dari A, yang dikembalikan oleh \textsl{Attribute\_selection\_method}
\end{enumerate}


\begin{figure}
\centering
\includegraphics[scale=1]{Gambar/jenishasilsplitpoint.jpg}
\caption[Jenis-jenis \textsl{split point}]{Jenis-jenis \textsl{split point}, Diterjemahkan dari \cite{DM}} 
\label{fig:splitPoint}
\end{figure}

Kemudian, algoritma \textsl{decision tree} akan dipanggil untuk setiap nilai hasil pembagian pada tuple, D$_{j}$  (baris 18).

Rekursif tersebut akan berhenti ketika salah satu dari kondisi terpenuhi, yaitu

\begin{enumerate}
	\item Semua tuple pada partisi D merupakan bagian dari kelas yang sama.
	\item Tidak ada atribut yang bisa dibagi (dilakukan pengecekan pada baris 4). Disini, akan dilakukan \textsl{majority voting} (baris 6) yang akan mengkonversi node N menjadi \textsl{leaf} dan diberi label dengan kelas yang terbanyak pada D.
	\item Sudah tidak ada tuple yang dapat diberi cabang, D$_{j}$ sudah kosong (baris 15) dan \textsl{leaf} akan dibuat dengan label bernilai \textsl{majority class} pada D (baris 16).
\end{enumerate}

Pada baris 21, akan dikembalikan nilai \textsl{decision tree} yang telah dibuat.

Terdapat beberapa teknik untuk memilih atribut pada \textsl{decission tree}, dua diantaranya adalah ID3 dan C4.5. ID3 merupakan teknik pemilihan atribut pada \textsl{decision tree} dengan memanfaatkan \textsl{entropy} dan \textsl{gain info} untuk menentukan atribut yang terbaik. Sedangkan C4.5 merupakan teknik lanjutan dari ID3 yang menggunakan \textsl{gain ratio} untuk melakukan pengecekan pada nilai \textsl{gain info}. Kedua teknik tersebut menggunakan pendekatan \textsl{greedy} yang merupakan \textsl{decission tree} yang dibangun secara \textsl{top-down recursive divide and conquer}. 

\paragraph{\textsl{Attribute Selection Measure}} merupakan suatu hirarki untuk pemilihan \textsl{splitting criterion} yang terbaik yang memisah partisi data (D) sesuai dengan tuple pelatihan kelas label ke dalam kelas masing-masing. \textsl{Attribute Selection Measure} menyediakan peringkat untuk setiap atribut pada training tuple. Jika \textsl{splitting criterion} merupakan nilai \textsl{continous} atau \textsl{binary trees}, maka nilai \textsl{split point} dan \textsl{splitting subset} harus ditentukan sebagai bagian dari \textsl{splitting criterion}. Contoh dari \textsl{attribute selection measure} adalah \textsl{information gain}, \textsl{gain ratio}, dan \textsl{gini index}.

Notasi yang digunakan adalah sebagai berikut. D merupakan data partisi, set pelatihan dari \textsl{class-labeled} tuple. Jika label kelas atribut memiliki m nilai yang berbeda yang mendifinisikan m kelas yang berbeda, $C_{i}$ (for i=1,...,m). C$_{i,d}$ menjadi kelas tuple dari C$_{i}$ di D. |D| dan |C$_{i,d}$| merupakan banyak tuple pada D dan C$_{i,d}$.

\subsubsection{ID3}

ID3 merupakan teknik untuk membuat \textsl{decision tree} dengan menggunakan \textsl{information gain} sebagai \textsl{attribute selection measure} untuk memilih atribut. Cara ID3 mendapatkan \textsl{information gain} dengan menggunakan \textsl{entropy}. \textsl{Entropy} adalah ukuran \textsl{impurity} (ketiadaan informasi) dari suatu data. Cara mendapatkan nilai \textsl{entropy} adalah

%\textsl{Information} menurut Claude Shannon dalam \textsl{information theory} adalah ukuran \textsl{pure} dari suatu data. Suatu data yang \textsl{pure} jika data tersebut memiliki tuple dengan \textsl{class} yang sama. ID3 menggunakan \textsl{information gain} sebagai \textsl{attribute selection measure} yang melakukan pemilihan atribut berdasarkan informasi yang terkandung dalam pesan. Cara ID3 mendapatkan \textsl{information gain} dengan menggunakan \textsl{entropy}. \textsl{Entropy} adalah ukuran \textsl{impurity} dari suatu data. Cara mendapatkan nilai \textsl{entropy} adalah

\begin{displaymath}
	Info(D) = -\sum_{i=1}^{m} p_{i} \log_2(p_{i})
\end{displaymath}

Dimana $p_{i}$ merupakan probabilitas tuple pada D terhadap class C$_{i}$, dapat diperoleh dengan |C$_{i,d}$|/|D|. Info(D) merupakan nilai rata-rata \textsl{entropy} dari suatu label kelas pada tuple D. Cara mengetahui atribut yang paling baik untuk dijadikan \textsl{splitting attribute} adalah dengan cara menghitung nilai \textsl{entrophy} dari suatu atribut kemudian diselisihkan dengan nilai \textsl{entropy} dari D. Jika pada tuple D, memiliki atribut A dengan v nilai yang berbeda, maka menghitung \textsl{entropy} dari suatu atribut adalah

\begin{displaymath}
	Info_A(D) = \sum_{j=1}^v \frac{|D_j|}{|D|} \times Info(D_j)
\end{displaymath}

|D$_{j}$|/D merupakan angka yang menghitung bobot dari suatu partisi.

Setelah mendapatkan nilai Info(D) dan Info$_{A}$(D), \textsl{information gain} dapat diperoleh dari selisih nilai Info(D) dan Info$_{A}$(D)

\begin{displaymath}
	Gain(A) = Info(D) - Info_A(D)
\end{displaymath}

Atribut yang memiliki nilai \textsl{gain information} yang terbesar akan dipilih sebagai output dari method ini.

contoh kasus untuk ID3, dalam pencarian \textsl{information gain}:

\begin{table}[h]
\centering
\caption{Contoh training set}
\label{table:contohTrainingSet}
\begin{tabular}{|l|l|l|l|l|l|}
\hline
RID & umur          & pendapatan 	& siswa 		 & resiko\_kredit  & Class: membeli\_komputer \\ \hline
1   & muda        	& tinggi   		& tidak      & cukup           & tidak                    \\ \hline
2   & muda        	& tinggi   		& tidak      & baik			       & tidak                    \\ \hline
3   & remaja 				& tinggi  	 	& tidak      & cukup           & ya                   \\ \hline
4   & dewasa      	& cukup 			& tidak      & cukup           & ya                   \\ \hline
5   & dewasa       	& rendah    	& ya    		 & cukup           & ya                   \\ \hline
6   & dewasa       	& rendah    	& ya    		 & baik			       & tidak                    \\ \hline
7   & remaja 				& rendah    	& ya    		 & baik			       & ya                   \\ \hline
8   & muda        	& cukup 			& tidak      & cukup           & tidak                    \\ \hline
9   & muda        	& rendah    	& ya    		 & cukup           & ya                   \\ \hline
10  & dewasa       	& cukup 			& ya   		   & cukup           & ya                   \\ \hline
11  & muda        	& cukup 			& ya   		   & baik 		       & ya                   \\ \hline
12  & remaja 				& cukup 			& tidak      & baik 		       & ya                   \\ \hline
13  & remaja 				& tinggi   		& ya    		 & cukup           & ya                   \\ \hline
14  & dewasa       	& cukup 			& tidak      & baik  			     & tidak                    \\ \hline
\end{tabular}
\end{table}

Pada table \ref{table:contohTrainingSet}, terdapat \textsl{training set}, D. Atribut kelas label merupakan dua nilai yang berbeda yaitu ya dan tidak, maka dari itu, nilai m = 2. C$_{1}$ diisi dengan kelas label bernilai ya, sedangkan C$_{2}$ diisi dengan kelas label bernilai tidak. Terdapat sembilan tuple atribut kelas label dengan nilai ya dan lima tuple dengan nilai tidak. Untuk dapat menentukan \textsl{splitting criterion}, \textsl{information gain} harus dihitung untuk setiap atribut terlebih dahulu. Perhitungan \textsl{entropy} untuk D adalah

\begin{displaymath}
	Info(D) = - \frac{9}{14}\log_2(\frac{9}{14}) - \frac{5}{14}\log_2(\frac{5}{14}) = 0.940 bits
\end{displaymath}

Setelah diperoleh nilai \textsl{entropy} dari D, kemudian akan dihitung nilai \textsl{entropy} atribut dimulai dari atribut umur. Pada kategori muda, terdapat dua tuple dengan kelas ya dan tiga tuple dengan kelas tidak. Untuk kategori remaja, terdapat empat tuple dengan kelas ya dan nol tuple dengan kelas tidak. Pada kategori dewasa, terdapat tiga dengan kelas ya dan dua dengan kelas tidak. Perhitungan nilai \textsl{entropy} atribut umur terhadap D sebagai berikut

\begin{align*}
	Info_{umur}(D) = \frac{5}{14} \times (-\frac{2}{5}\log_2\frac{2}{5} - \frac{3}{5}\log_2\frac{3}{5}) + \frac{4}{14} \times (-\frac{4}{4}\log_2\frac{4}{4} - \frac{0}{4}\log_2\frac{0}{4}) + \\
	\frac{5}{14} \times (-\frac{3}{5}\log_2\frac{3}{5} - \frac{2}{5}\log_2\frac{2}{5}) = 0.694 bits
\end{align*}

Setelah mendapatkan \textsl{entropy} dari atribut umur, maka nilai \textsl{gain information} dari atribut umur adalah

\begin{displaymath}
	Gain_{(umur)} = Info(D) - Info_{age}(D) = 0.940 - 0.694 = 0.246 bits
\end{displaymath}

Dengan melakukan hal yang sama, dapat diperoleh nilai \textsl{gain} untuk atribut pendapatan adalah 0.029 \textsl{bits}, untuk nilai \textsl{gain}(siswa) adalah 0.151 \textsl{bits}, dan \textsl{gain}(resiko\_kredit) = 0.048 \textsl{bits}. Karena nilai \textsl{gain} dari atribut umur merupakan nilai terbesar diantara semua atribut, maka atribut umur dipilih menjadi \textsl{splitting attribute}. Kemudian, node N akan membentuk cabang berdasarkan nilai dari atribut umur seperti pada gambar \ref{fig:hasilCabang}.

\begin{figure}
\centering
\includegraphics[scale=1]{Gambar/hasilcabangid3.jpg}
\caption[Hasil pohon faktor pada atribut \textsl{age} dari table 2.1]{Hasil cabang dari atribut \textsl{age}, Diterjemahkan dari \cite{DM}} 
\label{fig:hasilCabang}
\end{figure}

Untuk memilih atribut yang merupakan nilai kontinu, pencarian nilai \textsl{split point} harus dilakukan terlebih dahulu. Nilai yang diambil adalah nilai tengahnya untuk dijadikan \textsl{split-point}. Jika terdapat v nilai yang berbeda dari A, maka akan terdapat v-1 kemungkinan \textsl{split point}. Kemudian nilai \textsl{split point} akan dijadikan sebagai nilai pembagi, sebagai contoh: A <= \textsl{split-point} merupakan cabang pertama, dan A > \textsl{split-point} merupakan cabang kedua.

\subsubsection{C4.5}

\textsl{Information gain} akan memiliki nilai yang baik jika suatu atribut memiliki banyak nilai yang berbeda, namun hal itu tidak selalu bagus. Sebagai contoh kasus, jika nilai id suatu table yang memiliki nilai unik, maka akan terdapat banyak sekali cabang. Namun setiap cabang hanya akan berisi satu tuple dan bersifat \textsl{pure}, maka nilai \textsl{entropy} yang dihasilkan adalah 0. Oleh karena itu, informasi yang diperoleh pada atribut ini akan bernilai maksimum namun tidak akan berguna untuk \textsl{classification} \cite{DM}. Selain itu, ID3 dapat menghasil \textsl{decision tree} yang memprediksi secara berlebihan (\textsl{overestimated}) atau disebut juga \textsl{overfitting}. Hal ini dikarenakan pohon yang dihasilkan terlalu detail sehingga data input memiliki hasil prediksi yang pasti. 

C4.5 merupakan teknik lanjutan dari ID3, yang menggunakan \textsl{gain ratio} sebagai \textsl{attribute selection measure} untuk memilih atribut. Kemudian, C4.5 melakukan \textsl{tree pruning} untuk menghindari \textsl{overfitting}.

%\textsl{Information gain} akan memiliki nilai yang baik jika suatu atribut memiliki banyak nilai yang berbeda, namun hal itu tidak selalu bagus. Sebagai contoh kasus, jika nilai id suatu table yang memiliki nilai unik, maka akan terdapat banyak sekali cabang. Namun setiap cabang hanya akan berisi satu tuple dan bersifat \textsl{pure}, maka nilai \textsl{entropy} yang dihasilkan adalah 0. Oleh karena itu, informasi yang diperoleh pada atribut ini akan bernilai maksimum namun tidak akan berguna untuk \textsl{classification} \cite{DM}.

C4.5, menggunakan nilai tambahan dari \textsl{information gain} yaitu \textsl{gain ratio}, yang dapat mengatasi permasalahan \textsl{information gain} tentang nilai yang banyak. C4.5 melakukan teknik normalisasi terhadap \textsl{gain information} dengan menggunakan \textsl{split information} yang memiliki rumus sebagai berikut:

\begin{displaymath}
	SplitInfo_A(D) = - \sum_{j=1}^v \frac{|D_j|}{|D|} \times \log_2 (\frac{|D_j|}{|D|})
\end{displaymath}

Dimana |D| merupakan banyak data dan $|D_{j}|$ merupakan banyak data suatu nilai pada atribut.
Setelah mendapatkan nilai \textsl{split info} dari suatu atribut, dapat diperoleh nilai \textsl{gain ratio} dengan rumus sebagai berikut:

\begin{displaymath}
	GainRatio(A) = \frac{Gain(A)}{SplitInfo(A)}
\end{displaymath}

Nilai dari \textsl{gain ratio} terbesar yang akan dipilih. Perlu diketahui \cite{DM} jika nilai hasil mendekati 0, maka ratio menjadi tidak stabil, oleh karena itu, \textsl{gain information} yang dipilih harus besar, minimal sama besarnya dengan nilai rata-rata dari semua test yang diperiksa.

Contoh studi kasus, akan dilakukan perhitungan \textsl{gain ratio} dengan menggunakan training set pada table \ref{table:contohTrainingSet}. Dapat dilihat pada atribut pendapatan memiliki tiga partisi yaitu rendah, sedang, dan tinggi. Terdapat empat tuple dengan nilai rendah, enam tuple dengan nilai sedang, dan empat tuple dengan nilai tinggi. Untuk menghitung \textsl{gain ratio}, perlu dihitung nilai \textsl{split information} terlebih dahulu dengan cara:

\begin{displaymath}
	SplitInfo_A(D) = - \frac{4}{14} \times \log_2 (\frac{4}{14}) - \frac{6}{14} \times \log_2 (\frac{6}{14}) - \frac{4}{14} \times \log_2 (\frac{4}{14})
\end{displaymath} 
\begin{displaymath}
	SplitInfo_A(pendapatan) = 0.926 bits
\end{displaymath} 

Jika nilai \textsl{gain information} dari \textsl{income} adalah 0.029, maka, dapat diperoleh \textsl{gain ratio} dari pendapatan adalah

\begin{displaymath}
	GainRatio(pendapatan) = \frac{0.029}{0.926} = 0.031 bits
\end{displaymath}

Maka nilai \textsl{gain ratio} dari atribut pendapatan adalah 0.031 bits. Perhitungan tersebut dilakukan pada semua atribut, dan atribut yang memiliki nilai \textsl{gain ratio} yang terbesar adalah atribut yang dipilih.

\paragraph{Tree Pruning} merupakan proses pemotongan \textsl{decision tree} agar lebih efisien namun tidak terlalu mempengaruhi nilai keputusan yang dihasilkan. \textsl{decision tree} yang sudah dipotong akan lebih kecil ukuran pohonnya, tidak serumit dengan pohon yang asli, namun lebih mudah untuk diproses. \textsl{Decision tree} yang sudah dipotong memiliki kecepatan serta ketepatan mengklasifikasikan yang lebih baik \cite{DM}. Perbedaan \textsl{decision tree} yang sudah dipotong dan belum dapat dilihat pada gambar \ref{fig:treePruning}.

\begin{figure}
\centering
\includegraphics[scale=1]{Gambar/treepruning.jpg}
\caption[\textsl{Decision Tree Pruned}]{\textsl{Decision tree} yang belum dipotong dan yang sudah dipotong, Diterjemahkan dari \cite{DM}} 
\label{fig:treePruning}
\end{figure}

Terdapat dua pendekatan dalam melakukan \textsl{pruning}, yaitu \textsl{prepruning} dan \textsl{postpruning}.

\textsl{Prepruning}, pemotongan pohon dilakukan dengan cara menahan dan tidak melanjutkan pembuatan cabang atau partisi dari sebuah node, dan membuat node tersebut menjadi \textsl{leaf}. \textsl{Postpruning}, pemotongan pohon dilakukan ketika \textsl{decision tree} sudah selesai dibangun dengan cara menggubah cabang pohon menjadi \textsl{leaf}.

\subsection{\textsl{Pattern Evaluation}}
\textsl{Pattern evaluation} merupakan tahap mengidentifikasi apakah pola tersebut menarik dan merepresentasikan \textsl{knowledge} berdasarkan beberapa \textsl{interestingness measures}.
Suatu pola dapat dinyatakan menarik apabila
\begin{itemize}
	\item mudah dimengerti oleh manusia
	\item valid untuk data percobaan maupun data yang baru
	\item memiliki potensi atau berguna
	\item merepresentasikan \textsl{knowledge}
\end{itemize}

\subsection{\textsl{Knowledge Presentation}}
\textsl{Knowledge presentation} merupakan tahap representasi dan visualisasi terhadap \textsl{knowledge} yang merupakan hasil dari \textsl{knowledge discovery}. Hasil dari representasi dan visualisasi bisa dalam berbagai bentuk diantaranya adalah \textsl{flat data}, grafik, atau pohon keputusan.
%\section{\textsl{Spatial and Spatiotemporal}}

\section{Log Histori KIRI}

KIRI memiliki log histori yang melakukan pencatatan untuk setiap user ketika menggunakan KIRI. Data log tersebut diperoleh dengan cara melakukan wawancara dengan developer KIRI, yaitu Pascal Alfadian. Data log yang diberikan sudah dalam format excel.

\textsl{Log} tersebut memiliki 5 \textsl{field} untuk setiap tuple sebagai berikut:
\begin{itemize}
	\item logId, primary key dari tuple
	\item APIKey, mengidentifikasikan sumber dari pencarian ini
	\item \textsl{Timestamp} (UTC), waktu ketika pengguna KIRI mencari rute angkot, dalam waktu UTC / GMT
	\item \textsl{Action}, tipe dari log yang dibuat.
	\item AdditionalData, mencatat data-data yang berhubungan sesuai dengan nilai atribut\textsl{action}
\end{itemize}

LogId merupakan \textsl{field} dengan tipe data int dengan batas 6 karakter sebagai \textsl{primary key} dari table tersebut. LogId diisi dengan menggunakan fungsi \textsl{increment integer}. \textsl{Increment integer} merupakan fungsi untuk pengisian data pada database dengan menambahkan nilai 1 dari nilai yang terakhir kali diisi.
APIKey merupakan \textsl{field} dengan tipe data varchar untuk mengindentifikasi pengguna KIRI ketika menggunakan KIRI.
\textsl{Timestamp} (UTC) merupakan \textsl{field} dengan tipe data \textsl{timestamp} untuk mencatat waktu penggunaan KIRI oleh user, diisi dengan menggunakan fungsi \textsl{current time}. \textsl{Current time} merupakan fungsi untuk pengisian data pada database dengan mengambil waktu pada komputer ketika \textsl{record} dibuat.
\textsl{Action} merupakan \textsl{field} dengan tipe data varchar untuk memeriksa fungsi apa yang dipanggil dari API KIRI. Terdapat beberapa tipe pada \textsl{field action}, yaitu
\begin{itemize}
	\item \textsl{ADDAPIKEY}, \textsl{action} yang dicatat ke dalam log ketika fungsi pembuatan \textsl{API key} yang baru dipanggil.
	\item \textsl{FINDROUTE}, \textsl{action} yang dicatat ketika user melakukan pencarian rute
	\item \textsl{LOGIN}, \textsl{action} yang dicatat ketika developers melakukan login dengan menggunakan \textsl{API key}
	\item \textsl{NEARBYTRANSPORT}, \textsl{action} yang dicatat ketika user mencari transportasi di daerah rute sedang dicari
	\item \textsl{PAGELOAD}, \textsl{action} yang dicatat ketika user memasuki halaman KIRI
 	\item \textsl{REGISTER},\textsl{action} yang dicatat ketika developers melakukan pendaftaran pada KIRI \textsl{API key}
	\item \textsl{SEARCHPLACE}, \textsl{action} yang dicatat ketika user memanggil fungsi pencarian lokasi dengan menggunakan nama tempat
	\item \textsl{WIDGETERROR}, mencatat log tersebut ketika user menerima error dari \textit{widget}
	\item \textsl{WIDGETLOAD}, mencatat log tersebut ketika user melakukan download widget
\end{itemize}
AdditionalData, merupakan \textsl{field} dengan tipe data varchar untuk mencatat informasi sesuai dengan \textsl{field action}. Isi dari additionalData untuk setiap \textsl{action} adalah
\begin{itemize}
	\item Jika nilai atribut \textsl{action} adalah \textsl{ADDAPIKEY}, maka isi nilai dari additionalData adalah nilai \textsl{API key} yang dihasilkan
	\item Jika nilai atribut \textsl{action} adalah \textsl{FINDROUTE}, maka isi nilai dari additionalData adalah \textsl{latitude} dan \textsl{longitude} lokasi awal dan tujuan serta banyak jalur yang dihasilkan dari aplikasi KIRI
	\item Jika nilai atribut \textsl{action} adalah \textsl{LOGIN}, maka isi nilai dari additionalData adalah id dari user yang melakukan login serta status apakah user berhasil login atau tidak
	\item Jika nilai atribut \textsl{action} adalah \textsl{NEARBYTRANSPORT}, maka isi dari additionalData adalah \textsl{latitude} dan \textsl{longitude} dari transportasi tersebut
	\item Jika nilai atribut \textsl{action} adalah \textsl{PAGELOAD}, maka isi nilai dari additionalData adalah ip dari user
	\item Jika nilai atribut \textsl{action} adalah \textsl{REGISTER}, maka isi nilai dari additionalData adalah alamat email yang digunakan untuk meregister dan nama user
	\item Jika nilai atribut \textsl{action} adalah \textsl{SEARCHPLACE}, maka isi nilai dari additionalData adalah nama tempat yang dicari
	\item Jika nilai atribut \textsl{action} adalah \textsl{WIDGETERROR}, maka isi nilai dari additionalData adalah isi pesan dari error yang terjadi
	\item Jika nilai atribut \textsl{action} adalah \textsl{WIDGETLOAD}, maka isi nilai dari additionalData adalah ip dari user yang melakukan download widget
\end{itemize}

\section{\textsl{Haversine Formula}}\cite{Haversine}
\textsl{Haversine Formula} dapat menghasilkan nilai jarak antar dua titik pada bola dari garis bujur dan garis lintang titik tersebut. Berikut rumus Haversine:

\begin{displaymath}
	a = \sin^{2}((|\varphi_{1}-\varphi_{2}|)/2) + \cos\varphi_{1} . \cos\varphi_{2} . \sin^{2}((|\lambda_{1}-\lambda_{2}|)/2)
\end{displaymath}
\begin{displaymath}
	c = 2 . a\tan^{2}(\sqrt{a}, \sqrt{1-a})
\end{displaymath}
\begin{displaymath}
	d = R . c
\end{displaymath}

Dimana 
\begin{itemize}
	\item $\varphi$ adalah latitude dalam radian
	\item $\lambda$ adalah longitude dalam radian
	\item R adalah radius bumi (radius = 6,371km)
\end{itemize}

Contoh untuk perhitungan Haversine sebagai berikut:
Jika kita ingin menghitung jarak dua titik dari daerah Jakarta ke Surabaya, dengan titik pada Jakarta adalah -6.211544, 106.845172 dan titik pada Surabaya adalah -7.289166, 112.734398, maka perhitungan \textsl{haversine formula} akan menjadi

\begin{displaymath}
	a = \sin^{2}((|-6.211544-(-7.289166)|)/2) + \cos-6.211544 . \cos-7.289166 . \sin^{2}((|106.845172-112.734398|)/2)
\end{displaymath}
\begin{displaymath}
	a = 0.0026906745
\end{displaymath}

\begin{displaymath}
	c = 2 . 0.0026906745\tan^{2}(\sqrt{0.0026906745}, \sqrt{1-0.0026906745})
\end{displaymath}
\begin{displaymath}
	c =  0.1037900036
\end{displaymath}

\begin{displaymath}
	d = 6.371 X 0.1037900036
\end{displaymath}
\begin{displaymath}
	d = 0.6612461130 * 1000 km
\end{displaymath}
\begin{displaymath}
	d = 661.2461130 km
\end{displaymath}

Dengan menggunakan rumus Haversine, maka jarak antar kedua titik tersebut adalah 661.246 km

\section{Weka}\cite{Weka}
Weka merupakan aplikasi berbasis java yang berisi alat-alat untuk melakukan visualisasi dan algoritma data analisis serta pemodelan prediksi. Weka juga menyediakan file weka-src.jar yang berisi kelas-kelas yang dipakai oleh aplikasi weka sehingga user dapat menggunakannya untuk membuat program java yang berfungsi untuk \textsl{data mining}. Berikut beberapa kelas yang dimiliki oleh Weka:

\paragraph{\textsl{Classifier}} adalah sebuah \textsl{interface} yang digunakan sebagai skema untuk prediksi numerik ataupun nominal pada weka. 

\textsl{Constructor}:
\begin{itemize}
	
	\item void buildClassifier(Instances data)
	
	Method untuk melakukan klasifikasi dengan parameter set data pelatihan.
	
	\item double classfyInstance(Instances instance)
	
	Method untuk melakukan klasifikasi dari data dengan parameter data yang akan dilakukan klasifikasi. Method tersebut akan mengebalikan nilai kelas yang sesuai dengan data tersebut.
	
\end{itemize}

\paragraph{Instance} adalah interface yang mewakili set data.

\paragraph{Instances} adalah kelas untuk menangani set data.

\textsl{Constructor}:
\begin{itemize}
	\item Instances(java.io.Reader reader)
	
	Method constructor kelas \textsl{instances} yang menggunakan java.io.Reader untuk membaca file dengan format .arff. Data yang diterima dari file yang dibaca oleh kelas Reader akan langsung diubah menjadi kelas Instances dan disimpan pada objek Instances yang dibuat.
\end{itemize}

\paragraph{Attribute} adalah kelas yang digunakan untuk menangani atribut.

\paragraph{ID3} adalah kelas yang digunakan untuk membangun \textsl{decision tree} yang berbasis pada algoritma ID3, hanya dapat menerima input dengan atribut nominal.
\textsl{Method}:
\begin{itemize}
	\item void buildClassifier(Instances data)
	
	Membangun pohon keputusan dengan ID3 sebagai atribut\_method\_selection berdasarkan data input dalam kelas \textsl{Instances}.
	
	\item java.lang.String toString()
	
	Mengembalikan pohon keputusan yang sudah dibangun dalam bentuk String.
\end{itemize}

\paragraph{J48} adalah kelas yang digunakan untuk membuat \textsl{decision tree} c4.5.
\textsl{Method}:
\begin{itemize}	
	\item void buildClassifier(Instances instances)
	
	Membangun pohon keputusan dengan C4.5 sebagai atribut\_method\_selection berdasarkan data input dalam kelas \textsl{Instances}.
	
	\item java.lang.String toString()
	
	Mengembalikan pohon keputusan yang sudah dibangun dalam bentuk String.
	\end{itemize}

\paragraph{NumericToNominal} adalah kelas yang digunakan untuk mengubah nilai numerik menjadi nominal.
\textsl{Method}:
\begin{itemize}
	\item boolean setInputFormat(Instances instance)
	
	Mengubah input data yang akan diubah tipenya.
	
	\item void setOption(String[] option)
	
	Mengubah penyetingan pengaturan.
\end{itemize}

\section{Graphviz}\cite{Graph}
Graphviz merupakan perangkat lunak \textsl{open source} untuk visualisasi grafik. Dengan menggunakan graphviz, visualisasi grafik dapat dibuat dengan menulis kode. Bahasa pemograman yang digunakan oleh graphviz adalah bahasa DOT. DOT merupakan bahasa dekripsi grafik dalam bentuk \textsl{plain teks}. 

Pada bahasa DOT, pertama ditentukan bentuk grafik yang akan dibuat dengan cara menulis \textsl{graph} atau \textsl{digraph}. Grafik dapat diberi nama dengan cara menulis nama setelah penentuan bentuk grafik. Isi dari grafik akan diawali dengan tanda '{' dan diakhiri dengan tanda '}'. Contoh Penulisan untuk bentuk grafik serta penamaan grafik dapat dilihat pada listing \ref{lst:dotExample} baris 1. Berikut beberapa kata kunci yang merupakan isi dari grafik:

\begin{itemize}
	\item \textbf{\textsl{Node}}, merupakan kata kunci untuk membuat sebuah \textsl{node}. \textsl{Node} dapat dibuat dengan cara menuliskan nama node yang ingin dibuat. Kata kunci ini memiliki atribut yang dapat diubah, diantaranya adalah label \textsl{node}, bentuk \textsl{node}, ukuran \textsl{node}, warna \textsl{node}, dan \textsl{style node}. Penulisan atribut dapat diawali dengan tanda '[' dan diakhiri dengan tanda ']'. Contoh untuk penulisan \textsl{node} dapat dilihat pada listing \ref{lst:dotExample} baris ke 2,3, dan 5.
	\item \textbf{\textsl{Edge}}, merupakan kata kunci yang digunakan untuk mengubah atribut dari edge. Atribut \textsl{edge} yang dapat diubah diantaranya warna dari \textsl{edge} yang dibuat selanjutnya. Pengubahan atribut \textsl{edge} dapat dilakukan dengan cara menulis kata "edge" kemudian menulis atribut yang akan diubah yang diawali dengan tanda '[' dan diakhiri dengan tanda ']'. Contoh penulisan kata kunci \textsl{edge} dapat dilihat pada listing \ref{lst:dotExample} baris ke 6.
\end{itemize}

Untuk membuat sebuah \textsl{edge} dari node ke node yang lain, dapat dilakukan dengan cara menulis kedua nama node dan memberi tanda "->" diantara kedua node tersebut. \textsl{Edge} tersebut memiliki atribut yang dapat dapat diubah, salah satunya adalah label dari \textsl{edge}. Penulisan nilai atribut diawali tanda '[' dan diakhiri tanda ']' setelah penulisan nama node yang kedua. Contoh penulisan untuk membuat sebuah \textsl{edge} antara dua node dapat dilihat pada listing \ref{lst:dotExample} baris ke 4 dan 7.

Berikut contoh kode dengan bahasa DOT yang dapat dijadikan input untuk aplikasi graphviz: 

\begin{lstlisting}[caption={Dot Example}, label={lst:dotExample}]
	digraph G{
	Main
	Execute [shape=box, color=red, style=filled]
	Main -> Execute [label="proses dimulai"]
	Output [label="return result", width=2, height=1]
	edge [color=blue]
	Execute -> Output
	}
\end{lstlisting}

Maka hasil yang diperoleh dari perangkat lunak graphviz dapat dilihat pada gambar \ref{fig:outGraphviz}

\begin{figure}[H]
\centering
\includegraphics[scale=0.9]{Gambar/Graphviz.jpg}
\caption[Hasil output Graphviz]{Hasil output Graphviz} 
\label{fig:outGraphviz}
\end{figure}












