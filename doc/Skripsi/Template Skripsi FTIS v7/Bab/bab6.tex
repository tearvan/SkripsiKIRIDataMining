\chapter{Kesimpulan dan Saran}

\section{Kesimpulan}

Kesimpulan yang dapat diambil dari penelitian \textsl{data mining log} histori KIRI ini adalah 

Salah satu cara untuk memperoleh pola yang menarik dan bermakna, diperlukan pengolahan data dengan cara membuat klasifikasi dari data yang dihasilkan sesuai dengan tujuan yang ingin dicari. Pada penelitian ini dilakukan klasifikasi tujuan dari \textsl{user} apakah mereka ingin menuju Bandung atau keluar Bandung atau masih berada di area yang sama sehingga diperoleh pergerakan \textsl{user} KIRI di Bandung.

Pembuatan perangkat lunak untuk melakukan \textsl{data mining} pada \textsl{log} histori KIRI dapat dilakukan.

Pola yang diperoleh dari data \textsl{log} histori KIRI adalah \textsl{user} sering pergi menjauhi Bandung pada bulan Febuari 2014.

\section{Saran}
Untuk pengembangan aplikasi \textsl{data mining log} histori KIRI lebih lanjut, dapat dilakukan dengan cara menggunakan klasifikasi yang lebih baik dan detail. Perbedaannya dengan manggunakan klasifikasi yang lebih detail adalah hasil dari \textsl{decision tree} mungkin lebih besar (jika dibandingkan dengan \textsl{decision tree} yang dihasilkan dengan metode C4.5) namun diharapkan lebih memiliki makna dan dapat menghasilkan nilai akurasi yang lebih besar.

%Untuk pengembangan aplikasi \textsl{data mining log} histori KIRI lebih lanjut, dapat dilakukan dengan cara menggunakan klasifikasi yang lebih baik dan detail. Perbedaannya dengan manggunakan klasifikasi yang lebih detail adalah hasil dari \textsl{decision tree} belum tentu lebih kecil namun diharapkan lebih memiliki makna dan dapat menghasilkan nilai akurasi yang lebih besar.